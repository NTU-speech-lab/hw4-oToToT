\documentclass[a4paper,11pt]{article}
\usepackage[left=1.8cm, right=1.8cm, top=2.5cm, bottom=2cm]{geometry}
\usepackage{xeCJK}
\usepackage{hyperref}
\usepackage{indentfirst}
\usepackage{tabularx}
\usepackage{graphicx}
\usepackage{amsmath}
\usepackage{listings}
\usepackage{verbatim}
\usepackage{fancyhdr}
% \usepackage[compact]{titlesec}
\usepackage[usenames,dvipsnames]{xcolor}

\setCJKmainfont{NotoSansCJKtc-Thin}
\setmonofont{Consolas}

\definecolor{CodeGreen}{rgb}{0,0.6,0}
\definecolor{CodeGray}{rgb}{0.5,0.5,0.5}
\definecolor{CodeMauve}{rgb}{0.58,0,0.82}
\lstset{
    basicstyle = \ttfamily\footnotesize, 
    breakatwhitespace = false,
    breaklines = true,         
    commentstyle = \color{CodeGreen}\bfseries,
    extendedchars = false,
    keepspaces=true,
    keywordstyle=\color{blue}\bfseries, % keyword style
    language = C++,                     % the language of code
    otherkeywords={string},
    numbers=none,
    numbersep=5pt,
    numberstyle=\tiny\color{CodeGray},
    rulecolor=\color{black},
    showspaces=false,
    showstringspaces=false,
    showtabs=false,
    stepnumber=1,       
    stringstyle=\color{CodeMauve},        % string literal style
    tabsize=2,
}
% from https://blog.csdn.net/RobertChenGuangzhi/article/details/45126785

\title{Machine Learning 2020 - Homework 3 Report}
\author{學號:b08902100, 系級:資工一, 姓名:江昱勳}
\date{}

\begin{document}
\pagestyle{fancy}
\fancyhead[L]{Machine Learning 2020 - Homework 6}
\fancyhead[R]{Author: b08902100 江昱勳}

\maketitle

% \verbatiminput{HW2_S.txt}
% \lstinputlisting{HW2.cpp}

\begin{enumerate}

\item 請說明你實作的RNN的模型架構、word embedding 方法、訓練過程(learning curve)和準確率為何?

我的RNN模型架構由三層雙向的LSTM加上三層的全連接層組成,每層LSTM後接帶有機率0.5的Dropout層,而每層全連階層之間的活化函數則是使用ReLU,而最後會將輸出通過Sigmoid函數讓數值在0~1之間。word embedding的部份使用gensim內建的word to vector模型做訓練,嘗試將文字輸出成500維的向量,使用skip-gram方式並且將window大小設定成5,而一個字要出現超過5次才會將其考慮進來,最後訓練25個epoch。

RNN的訓練過程可以參考下面的圖片

\includegraphics[width=.45\textwidth]{acc.pdf}
\includegraphics[width=.45\textwidth]{loss.pdf}

最後我是將validation accuracy最高的model存下來使用,其準確率(kaggle上的public score)為0.82232。


\item 請比較BOW+DNN與RNN兩種不同model對於``today is a good day, but it is hot''與``today is hot, but it is a good day''這兩句的分數(過softmax後的數值),並討論造成差異的原因。

我建立了一個簡單的DNN,總共由5層全連接層組成,除了最後一層以外,接以ReLU作為活化函數連接,最後一層則透過sigmoid函數將數值控制在0~1之間,而BOW的部份我則僅統計總出現次數超過10次的單字,並刪除一些標點符號以及停用詞來讓構成的向量大小不會太大;該model在kaggle上的public score為0.79189。

以BOW+DNN模型預測``today is a good day, but it is hot''與``today is hot, but it is a good day''的分數為0.76685613,而RNN則輸出0.2985。可以發現RNN將該句話視為叫為負面的語句,而BOW+DNN則相反,推測可能是因為BOW+DNN沒有整段話的資訊,無法分辨出but所代表的轉折語氣,而RNN因為會完整有順序的將一個句子閱讀完,因此叫BOW+DNN能體現語句中的轉折。

\item 請敘述你如何 improve performance(preprocess、embedding、架構等等),並解釋為何這些做法可以使模型進步,並列出準確率與improve前的差異。

我有參考過 ``Pedro M. Sosa, (2017, June 7), \textit{Twitter Sentiment Analysis using combined LSTM-CNN Models} [Online]. Available: \url{https://www.academia.edu/35947062/Twitter_Sentiment_Analysis_using_combined_LSTM-CNN_Models}''描述的方法,將LSTM搭配CNN構築model,然而我並沒有看到模型的進步。我也嘗試將LSTM改成GRU過,其performance與LSTM無太大的差異(在kaggle上的public score都是0.82232)。而嘗試換過word to vector的維度以及訓練次數皆沒有顯著的進步。

\item 請描述你的semi-supervised方法是如何標記label,並比較有無semi-supervised training對準確率的影響並試著探討原因。

% \begin{figure}[ht]
%     \center
%     \includegraphics[width=.8\textwidth]{confusion.pdf}
%     \caption{confusion matrix}
%     \label{fig:conf}
% \end{figure}

% 從圖\ref{fig:conf}中可以看到,這個model最容易將3跟0搞混,1跟2之間也很容易被搞混。

\end{enumerate}

\end{document}